\documentclass[12pt]{article}
% We can write notes using the percent symbol!
% The first line above is to announce we are beginning a document, an article in this case, and we want the default font size to be 12pt
\usepackage[utf8]{inputenc}
% This is a package to accept utf8 input.  I normally do not use it in my documents, but it was here by default in Overleaf.
\usepackage{amsmath}
\usepackage{amssymb}
\usepackage{amsthm}
% These three packages are from the American Mathematical Society and includes all of the important symbols and operations 
\usepackage{fullpage}
% By default, an article has some vary large margins to fit the smaller page format.  This allows us to use more standard margins.

\setlength{\parskip}{1em}
% This gives us a full line break when we write a new paragraph


\begin{document}
% Once we have all of our packages and setting announced, we need to begin our document.  You will notice that at the end of the writing there is an end document statements.  Many options use this begin and end syntax.

\begin{flushright}
    Mateusz Biegański \\ 
    mb385162
\end{flushright}

\begin{center}
    \Large Computational Complexity - Homework 3 \normalsize
\end{center}


\textbf{4 minutes of delay, I would be grateful if I weren't punished for that :)}:

\textbf{Problem 3.1}:
Prove that there exists a deterministic Turing machine with oracle for
SAT that works in polynomial time, and that given a positive integer n finds its decomposition into prime numbers:
$n = p^{\alpha_1} \cdot...\cdot p^{\alpha_k} $ where $p_1 < ... < p_k$ are prime numbers, and $\alpha_1,...,\alpha_k$ are positive integers.


{\bf Solution}:

Let's observe, that for given $n, n\in \mathbb{N}$, SAT problem can be used to find it's decomposition into positive numbers $x, y$, such that $n = x \cdot y$. That construction is described in Appendix.

Now, we simply do the following: 
Start an algorithm from $n$; check whether it is prime, using AKS algorithm. If it is, finish, otherwise find it's $n = x\cdot y$ decomposition, using algorithm described in Appendix. Now, do the same for both it's $x, y$ factors.

This way, we obtain decomposition $n = p_1 \cdot p_2 \cdot ... \cdot p_m, \forall{i \in \{1, ..., m\} }$ $p_i$ is prime. Now, to find all $\alpha_i$ numbers from problem description we just sort all $p_i$ and group it.

Whole solution is polynomial-time in obvious way, thus we proved such TM exists.


\textbf{Appendix:}

Let's consider problem of determining decomposition of given $n \in \mathbb{N}$ into factors $x,y$, such that $x, y \geq 2, n = x \cdot y$ using SAT. We can simply omit that part, because SAT is NP-complete and finding multiplication decomposition is in NP, but I put that part for consistency, rather as a draft.

Let's say that $n$ is given in binary representation $z_k,...,z_2,z_1$, $z_i$ representing each digit, $1$ index means the lowest significant bit.
We will be using variables $x_k, ..., x_1$ and $y_k, ..., y_1$ to mark representation of $x, y$ numbers. Let's observe, that multiplication of binary numbers consists of addition of maybe shifted numbers, which is easy to obtain if we implement addition.

$z_i \implies XOR(x_i, y_i, z_{i-1}),$ assuming $z_0 = 0$ implements addition.

Now, for multiplication 10101 * 1101, we do
10101(0) + 10101(2) + 10101(3) (number in parentheses means positions to shift left)
= 10101000 + 1010100 + 10101
Remark, that we take advantage of fact, that XOR is implementable in boolean formulas language.

Now, when we do have machine determining whether formula is satisfiable, we can create polynomial-time algorithm that finds proper values ($x,y$ in our case), which just substitutes $x_i = 1$, and check whether formula is still satisfiable, if no, it say's "then $x_i$ must be 0" and do the same for $x_{i+1}$.
\\


We need to remember that we only need to find factors $\geq 2$, so in our formula we must have additional clauses that forces any of $x_1, ..., x_{k-1}$ being 1, same with $y_1, ..., y_{k-1}$.


Whole description above provides us algorithm that for given $n$ finds such $x, y$, that $n = x \cdot y$. 



\textbf{Problem 3.2}:

For a language $L \subseteq \{0,1\}\ast$, let 
$B(L, r) = \{u | \exists{}v \in L, d(u, v) \leq r\}$, where $d(u, v)$ is the Hamming distance,
  \begin{equation*}
    d(u, v) =
    \begin{cases}
      $\rvert \{ i : v_i \neq u_i \} \rvert$ , & $ \rvert u \rvert = \rvert v \rvert$ \\
      $\infty$, & \text{otherwise}
    \end{cases}
  \end{equation*}
  
Show that for each $L \subseteq \{0, 1\}$ and each $r \in \mathbb{N} $

\textbf{3.2 a)}:
\text{if $L \in RP$ then $B(L, r) \in RP$}
Let $M$ be RP-TM that recognizes $L$. 

To show $B(L, r)$ is in RP, for given word $w$ we must construct RP-TM. Let's observe, that we seek for words $u \in L$, such that $d(w, u) \leq r$. Thus words $u$ are copies of $w$, with $i \in \{0,1,...,r\}$ bits changed. There are $\sum_{i=0}^{i = r} {\binom{n}{i}} \leq n^{r+1} = poly(n) $ such words.
What we do is simply run $M$ for each word from above, waiting for TRUE answer.
Let's observe that it is enough for RP-TM construction - if $w \notin B(L, r)$, we never return TRUE; if it is in $B(L,r)$, in particular $\exists{u}$ $d(u, w) \leq r$, and we run $M$ for that $u$, obtaining required $1/2$ success chance.



\textbf{3.2 b)}:
\text{if $L \in coRP$ then $B(L, r) \in coRP$}

$L \in coRP$ implies, that there exists coRP-TM, recognizing $L$. Let's name it M. We must show, that there exists coRP-TM recognizing $B(L, r)$. 
The computation is presented below:

Given word $w, \rvert w\rvert = n$ let's mark $W = \sum_{i = 0}^{i \leq r} {\binom{n}{i}}$ (number of all considered words).

First generate all words $v$, such that $d(w,v) \leq r$.
Now, run $M$ on all such generated $v's$, accept if any was accepting. But, what is important, on each word run $M$ machine $ W$ times (once is not enough; it will be needed to prove correctness). We accept if all of $W$ runs are accepting. 

Let's observe, that for words from $B(L, r)$ at least one word $v$ is from $L$, thus we will accept with probability equals to 1.

A bit more problematic is case, when $w \notin B(L, r)$. $M$ rejects $w$ with probability $p \geq 1/2$. We need to compute probability, that $\forall{v, d(w,v) \leq r}$ M rejects $v$ $W$ times. Probability of single $v$ to be rejected during $W$-times-check is $1 - (1-p)^W$ (it is enough one run to be rejecting), thus rejection of all $v's$ is $(1 - (1 - p)^W)^W \geq (1 - 2^{-W})^W \geq 1/2$, because $p \geq 1/2$.

Algorithm above belongs to coRP.


\end{document}

