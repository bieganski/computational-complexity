\documentclass[12pt]{article}
% We can write notes using the percent symbol!
% The first line above is to announce we are beginning a document, an article in this case, and we want the default font size to be 12pt
\usepackage[utf8]{inputenc}
% This is a package to accept utf8 input.  I normally do not use it in my documents, but it was here by default in Overleaf.
\usepackage{amsmath}
\usepackage{amssymb}
\usepackage{amsthm}
% These three packages are from the American Mathematical Society and includes all of the important symbols and operations 
\usepackage{fullpage}
% By default, an article has some vary large margins to fit the smaller page format.  This allows us to use more standard margins.

\setlength{\parskip}{1em}
% This gives us a full line break when we write a new paragraph


\begin{document}
% Once we have all of our packages and setting announced, we need to begin our document.  You will notice that at the end of the writing there is an end document statements.  Many options use this begin and end syntax.

\begin{flushright}
    Computational Complexity \\
    Mateusz Biegański \\ 
    mb385162
\end{flushright}

\begin{center}
    \Large Homework 2b \normalsize
\end{center}

\textbf{Question}:
Consider words of the form $w_1w_2 ...w_{2m}$, where all $w_i$ are words of
length m over the alphabet $\{0, 1\}$. Let $Perm$ be the set of those words of this form in which the
words $w_i$ are pairwise different (i.e., these are permutations of all m-bit numbers). Prove that
$Perm$ belongs to log-space uniform $AC0$.



{\bf Solution}

Let $n$ be the input size ($n = m \cdot 2^m$). I construct circuit described below, that recognizes language $Perm$.

First, our TM ensures us that length of input word $1^n$ equals $k\cdot2^k$ for any $k \in N$ and returns that $k$. It can be done in $log(n)$ space (in appendix I add algorithm for similar computation).
Having $k$ (equals to $m$ from task description), we construct circuit as presented below:

Let $S_n$ be the set of all words of length $n$ over alphabet $\{0,1\}$.

For word $w$ let's define as $C_w$ circuit, that for input $v$, such that $|v| = |w|$ returns 1 if and only if $w = v$. It can be simply obtained by $AND$ gate of $|w|$ fanin over $XNOR$ gates per each letter. Depth of such circuit is $O(1)$.

For each input word $w$ of length $k\cdot2^k$ let's define $I_{w}$ as the set of all well-formed infixes. Well-formed means here, that $I_w$ contains only infixes of length $k$, starting at positions $\{0, k, 2k, ... , 2^k - k\}$ (simply, number that maybe are permutations).

Let's say we generate circuit for word $w$, $|w| = m\cdot2^m$.
For each infix from $I_w$ we generate circuit $C_v$ for every word $v$ from $S_{|v|}$.

Let's observe, that we obtain $2^m \cdot 2^m = O(n^2)$ results.


\textbf{Appendix:}

keep counter $k$, equals to 0 at the beginning and counter $i = 1$. Each iteration do the following:
Starting from cell 0, go $2*i - 1$ cells right. Increase $k$ by 1. If head is over 1, and on it's right there is $blank$, then return $k$. If head is over blank, throw error (input word's length is not in $2^k$ form for any k). If head is over 1, go to next iteration.

\end{document}

